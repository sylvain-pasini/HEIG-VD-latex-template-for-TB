
\chapter{Planification}
\label{ch:planif}

\section{Planification initiale}
Un exemple de planification en latex. D'autres moyens sont possibles.

%exemple
\scalebox{0.6}{
\begin{ganttchart}[vgrid,
title/.style={fill=teal, draw=none},
title label font=\color{white}\bfseries,
title left shift=0,
title right shift=0,
%title top shift=0,
title height=.75,
group/.append style={draw=black, fill=black!50},
%group top shift=1,
bar/.append style={fill=gray!50},
%bar top shift=0,
%bar height=0.4,
%milestone top shift=0,
y unit chart=0.6cm,
x unit=0.6cm,
]{1}{28}
\gantttitle{Déroulement du projet}{28} \\
\gantttitlelist{1,...,28}{1} \\
%
\ganttgroup{Démarrage}{1}{5} \\
	\ganttbar{Kick-off meeting}{1}{1} \\
	\ganttbar{Documents administratifs (NDA, etc.)}{1}{2} \\
	\ganttbar[bar/.append style={fill=red!50}]{Analyse des besoins métiers}{2}{4} \\
	\ganttbar{Rédaction du cahier des charges}{2}{4} \\
%	
\ganttgroup{Réalisation de l'état de l'art}{3}{6} \\
	\ganttbar{Recherche des travaux}{3}{3} \\
	\ganttbar[bar/.append style={fill=red!50}]{Analyse du travail précédent}{4}{5} \\
	\ganttbar{Analyse de la techno X}{5}{6} \\
%	
\ganttgroup{Mise en place de l'environnement de dev}{5}{8} \\
	\ganttmilestone[milestone/.append style={fill=red!50}]{Liste des besoins}{5}{12} \\
	\ganttbar{Mise en place}{7}{8} \\
%
\ganttgroup{Développement}{9}{19} \\
	\ganttbar{Back-end}{9}{15} \\
	\ganttbar{Front-end}{14}{17} \\
	\ganttbar{Prod}{18}{19}\\
%
\ganttbar{Tests}{19}{21} \\
%
\ganttbar{Synthèse des résultats}{20}{22} \\
%
\ganttgroup{Documentation}{1}{22} \\
	\ganttmilestone[milestone/.append style={fill=red!50}]{Rapport intermédiaire}{12}{12} \\
	\ganttmilestone[milestone/.append style={fill=red!50}]{Rapport final}{22}{22} \\
	\ganttmilestone[milestone/.append style={fill=red!50}]{Défense}{28}{28} \\
%
\end{ganttchart}
}

